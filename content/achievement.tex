\chapter{攻读学位期间的成果}

\vspace{8pt}
%% 盲审版本匿名化成果
% \begin{enumerate}[itemsep=0.7em, leftmargin=2em, itemindent=0em,labelwidth=2em]
% 	\item[1.] 中文信息学报. 已录用. 第一作者
% 	\item[2.] 中文信息学报. 已录用. 第一作者
% 	\item[3.] Proceedings of the 2021 Conference of the International Conference on Asian Language Processing (IALP), 2021 .第一作者
% 	\item[4.] Proceedings of the 2022 Conference of the North American Chapter of the Association for Computational Linguistics (NAACL), 2022. 已投出. 第一作者
% 	\item[5.] 中文信息学报. 已录用. 第三作者
% 	\item[6.] Proceedings of the 44th European Conference on Information Retrieval (ECIR), 2022. 第三作者
% 	\item[7.] 中文信息学报, 2021. 第三作者
% 	\item[8.] Proceedings of the 2022 Conference of the International Joint Conference on Neural Networks (IJCNN), 2022. 已投出. 第三作者
% \end{enumerate}

% 盲审版本匿名化成果
% \begin{enumerate}[itemsep=0.7em, leftmargin=2em, itemindent=0em,labelwidth=2em]
% 	\item[1.] Proceedings of the 44th European Conference on Information Retrieval (ECIR), 2022. 第一作者 (\textbf{CCF C})
% 	\item[2.] Proceedings of the 44th European Conference on Information Retrieval (ECIR), 2023. 第一作者 (\textbf{CCF C})
% 	\item[3.] Proceedings of the 60th Annual Meeting of the Association for Computational Linguistics (ACL), 2023. 已投出 第一作者 (\textbf{CCF A})
% 	\item[4.] 中文信息学报. 已录用 第一作者 (\textbf{核刊}) 
% 	\item[5.] Proceedings of the 2021 Conference of the International Conference on Asian Language Processing (IALP), 2021 .第二作者 \textbf{EI} 
% 	\item[6.] 中文信息学报. 已录用. 第二作者 (\textbf{核刊}) 
% 	\item[7.] Proceedings of the 44th European Conference on Information Retrieval (ECIR), 2023. 第三作者 (\textbf{CCF C})
% 	\item[8.] 山西大学学报. 已录用. 第四作者 (\textbf{核刊}) 
% \end{enumerate}

%% 非盲审版本成果
% \begin{enumerate}[itemsep=0.7em, leftmargin=2em, itemindent=0em,labelwidth=2em]
% 	\item[1.] \textbf{朱鸿雨},洪宇,苏玉兰,唐竑轩,尉桢楷,张民. 基于多粒度交互推理的答案选择方法研究. 中文信息学报. 已录用.(\textbf{核刊})
% 	\item[2.] \textbf{朱鸿雨},金志凌,苏玉兰,张民,洪宇. 面向问题复述识别的定向数据增强方法. 中文信息学报. 已录用.(\textbf{核刊})
% 	\item[3.] \textbf{Hongyu Zhu}, Yan Chen, Jing Yan, Jing Liu, Yu Hong, Ying Chen. CQM$_{robust}$: A Chinese Dataset of Linguistically Perturbed Natural Questions for Evaluating the Robustness of Question Matching Models. EMNLP 2022.待投出. (\textbf{CCF B})
% 	\item[4.] \textbf{Hongyu Zhu}, Zhiling Jin, Yan Bai, Yu Hong, Min Zhang. Multi-granularity Interaction Fusion for Neural Answer Source Selection[C]//Proceedings of the 2021 Conference of the International Conference on Asian Language Processing (IALP). IEEE, 2021: 13-18. (\textbf{EI})
% 	\item[5.] 苏玉兰,洪宇,\textbf{朱鸿雨},武恺莉,张民. 基于大规模问答数据的噪声感知问题生成. 中文信息学报. 已录用.(\textbf{核刊})
% 	\item[6.] Zhiling Jin, Yu Hong, \textbf{Hongyu Zhu}, Jianmin Yao, Min Zhang. Bi-granularity Adversarial Training for Non-Factoid Answer Retrieval[C]//Proceedings of the European Conference on Information Retrieval. Springer, Cham, 2022: 322-335. (\textbf{CCF C})
% 	\item[7.] 武恺莉,朱朦朦,\textbf{朱鸿雨},张熠天,洪宇. 结合问题类型及惩罚机制的问题生成[J]. 中文信息学报, 2021, 35(4): 110-119.(\textbf{核刊})
% 	\item[8.] Yulan Su, Yu Hong, \textbf{Hongyu Zhu}, Minhan Xu, Yifan Fan. Binary-perspetive Asymmetrical Twin Gain: a Novel Evaluation Method for Question Generation. IJCNN 2022. 已录用. (\textbf{CCF C})
% \end{enumerate}

% 非盲审版本成果
\begin{enumerate}[itemsep=0.7em, leftmargin=2em, itemindent=0em,labelwidth=2em]
	\item[1.] \textbf{Zhiling Jin}, Yu Hong, Hongyu Zhu, Jianmin Yao, Min Zhang. Bi-granularity Adversarial Training for Non-factoid Answer Retrieval[C]//Advances in Information Retrieval: 44th European Conference on IR Research, ECIR 2022, Stavanger, Norway, April 10–14, 2022, Proceedings, Part I. Cham: Springer International Publishing, 2022: 322-335. (\textbf{CCF C})
	\item[2.] \textbf{Zhiling Jin}, Yu Hong, Rui Peng Jianmin Yao, Guodong Zhou. Intention-aware Neural Networks for Question Paraphrase Identification. ECIR 2023 已录用. (\textbf{CCF C})
	\item[3.] \textbf{金志凌},朱鸿雨,苏玉兰,唐竑轩,洪宇,张民. 基于多粒度交互推理的答案选择方法研究. 中文信息学报. 已录用.(\textbf{核刊})
	\item[4.] \textbf{Zhiling Jin}, Yu Hong, Zhifeng Li, Jianmin Yao, Guodong Zhou. GBT: Generation Boosting Training for Sentence Semantic Matching. ACL 2023 已投出. (\textbf{CCF A})
	\item[5.] Hongyu Zhu, \textbf{Zhiling Jin}, Yan Bai, Yu Hong, Min Zhang. Multi-granularity Interaction Fusion for Neural Answer Source Selection[C]//Proceedings of the 2021 Conference of the International Conference on Asian Language Processing (IALP). IEEE, 2021: 13-18. (\textbf{EI})
	\item[6.] 朱鸿雨,\textbf{金志凌},苏玉兰,张民,洪宇. 面向问题复述识别的定向数据增强方法. 中文信息学报. 已录用.(\textbf{核刊})
	\item[7.] Rui Peng, Yu Hong, \textbf{Zhiling Jin}, Jianmin Yao and Guodong Zhou. Feature Differentiation and Fusion for Semantic Text Matching. ECIR 2023 已录用. (\textbf{CCF C})
	\item[8.] 李志峰,邹博伟,李烨秋,\textbf{金志凌},洪宇. 基于多知识源融合的级联式常识问答方法. 山西大学学报. 已录用. (\textbf{核刊})
\end{enumerate}