\chapter{绪论}

本章将对问答匹配任务展开详细介绍。
首先,本章阐述了问答匹配任务的研究背景及意义。
% 其次,本章介绍了该任务的国内外研究现状。
接着,本章总结了前人的科研工作,分别从基于表示、交互和数据增强三个方面,详细介绍了国内外研究现状。
然后,本章分析并总结了问答匹配领域现有的科学问题及研究难点,并针对以上问题和难点,阐述了本文的主要研究内容。
最后,本章简述了全文的组织架构。

\section{研究背景}

%% 问答系统的重要性
% 现有互联网信息获取的主要渠道为搜索引擎,其根据用户的提问返回一系列相关的网页链接。
% 人们通常在互联网上获取信息的主要渠道为搜索引擎,其根据用户的提问返回一系列相关的网页链接。
% 近年来,互联网技术高速发展,网络数据也呈爆炸式增长,
% 这导致用户需要花费大量的时间与精力从繁杂冗余的搜索结果中获取有用信息。
% 同时,网络上数据的质量参差不齐,传统搜索引擎无法有效甄别数据的合理性。
% 在此背景下,智能问答系统应运而生。

%% 问答系统的细分及匹配技术的重要性
问答匹配(Question Answer Matching)是自然语言处理(Natural Language Processing,简称NLP)领域重要的研究方向。
目前已在多种场景,如问答系统、客服机器人、语音助手等,得到广泛应用,当前主流的搜索引擎(百度、谷歌等)也已引入基于深度学习的问答匹配技术,以提供更加精准可靠的搜索结果。
问答匹配任务可以通俗地理解为面向问答场景文本的语义匹配任务。
作为问答匹配的子任务,答案选择(Answer Selection)以及问题复述识别(Question Paraphrase Identification)从语义匹配的不同角度为实际应用提供服务。
答案选择任务需要模型根据给定问题与候选答案的相似性,对候选答案进行排序,选择出相关性较高的答案进行返回。
问题复述识别任务则需要模型识别问句对子之间的语义一致性,即两个问句的意思是否一致,从而从已知答案的庞大问答对子中找出与用户提问最相关的问题,并返回其答案。
从答案选择以及问题复述识别任务的定义可知,两者分别从“问题与答案”的语义相关性以及“问题与问题”的语义一致性的两个角度,为智能问答系统,搜索引擎等实际应用提供支持。
其对知识获取、信息检索等领域也具有重要影响力。因此,问答匹配技术的研究具有十分重要的意义。

% 问答匹配的发展(预训练) 及缺点
随着预训练语言模型的发展,其优秀的特征提取和语义编码能力在NLP领域带来了显著的提升,其中也包括问答匹配任务。
问答匹配技术的研究一直备受关注,基于预训练语言模型的深度学习方法也成为了问答匹配任务的主流方法。
但是尽管研究的方法不断改进,却始终存在着两个挑战性的问题:其语义表征能力难解困难样例和问句意图感知能力弱。
首先,困难样本难解的根本原因是模型的语义表征不准确,没有精细地感知到语句中细微且关键的要素。
利用困难样本生成技术,例如对抗攻击方法,构造具有迷惑性的样本,加入模型的训练,是提高语义感知和表征能力的有效方法。
研究如何高效且高质量地生成困难样本对于提高问答匹配模型性能具有着重要的意义。
其次,问句意图感知能力弱也是当前问答匹配技术需要解决的另一个关键问题。
对于问句来说,意图信息是十分关键的。
例如:“您今晚什么时候吃饭?”和“您今晚什么地方吃饭?”,尽管字面十分相似,但是前者的意图是对时间提问,而后者是对地点提问。
仅仅凭借预训练语言模型的方法容易被相近的字面所迷惑,而忽视了更加关键的意图信息。
这些问题目前还缺少有效的解决方法。


% 但是,这并不意味着模型在实际应用中仍能保持其高性能。
% 在真实的语言场景下,一些简单且易于回答的问题,最前沿的模型却会给出错误回答。
% 在实际应用中,问答匹配技术仍处于起步阶段,需要不断革新,使问答系统更加高效、准确。

\section{研究意义}
问答匹配包括答案选择以及问句复述识别两个子任务。问答匹配无论是对实际业务应用或是自然语言处理的科学研究都有着重要的研究意义。
此处从以下三个方面阐述其研究意义:

\textbf{\songti(1)广泛的实际业务应用}

问答匹配任务有着广泛的实际业务需求应用,例如搜索引擎、自动问答、智能客服等领域。
其需要系统能够充分理解用户所输入问句的语义,并从已有的数据源中匹配出相关性最佳的结果。
现有的主流搜索引擎中已经引入基于深度学习的问答匹配技术。
用户输入问题后,搜索引擎内部利用问答匹配和机器阅读理解等技术,解析出精确的答案作为第一条结果返回,极大提高用户搜索体验。
此外,例如信息检索技术可以归结为是用户输入的查询与文档资源的语义匹配,自动问答系统则是通过用户查询与已有候选答案或者是已有“问答对子”的问句进行匹配。
包括众多商家如淘宝、京东等都接入了智能客服系统,用户解决海量的客户问题及反馈。
利用这种方式可以高效的解决用户的常见问题,降低运营成本,提升用户体验,这也是问答匹配技术的一项成熟的应用。
问答匹配是自然语言理解中的核心问题,其泛用性成为许多前沿智能系统的基础技术。


\textbf{\songti(2)对问答系统的重要意义}

传统的搜索引擎根据用户查询返回相关的网页,这样的方式导致用户无法快速便捷地获取关键信息。
而由深度学习技术兴起的自动问答系统以“一问一答”的智能形式,给予用户良好的交互体验。
通过这种方式可以精确定位用户意图,从而快速高质量地满足用户的信息需求,达到降低成本,提高用户体验的效果。
答案选择技术提供了“问题-答案”相关性语义的支持,根据用户所输入的问句的语义信息,从已有的候选答案数据中匹配出最佳结果。
例如,该技术应用于基于知识库的问答系统(Knowledge Base Question Answering, 简称KBQA),为用户输入的问题召回相关性最高的知识,以用于模型结合问句以及知识之后产出答案。
问题复述识别则从“问题-问题”之间的语义一致性出发。根据用户输入的目标问句,从已知答案的候选“问句-答案”对子中,匹配出语义相同或高度相似的复述问句,将其答案返回给用户。
例如对于常用的检索式问答系统,以及主流的搜索引擎(如百度,谷歌等)而言,问句复述识别是实现该类系统的关键。
因此,提高问答匹配技术对于自动问答,信息检索等智能系统的准确性和可靠性的具有非常重要的意义。


\textbf{\songti(3)为自然语言处理研究提供支持}

问答匹配任务可以通俗地理解为面向问答场景文本的语义匹配任务,其任务的核心在于理解句子对的语义,并且根据任务场景作出相应的判断。
其语义理解和匹配的能力为自然语言处理领域的其他任务提供了强大的支持。
例如,在常识问答(Commonsense QA)任务中,需要模型在获取到问题之后,检索与问题相关的常识知识。借由这些知识的语义交互,模型才能产出该问题的答案。
问答匹配技术能够根据问题,有效地检索出知识库中与之相关的常识信息,返回给模型。问答匹配技术的召回的知识是否准确和可靠,也成为了常识问答模型是否能够准确回答问题的关键。
在开放域阅读理解(Open-domain Machine Reading Comprehension)任务中,首先需要借助问答匹配技术,在庞大的文档资源中,匹配出较为相关的段落,继而在匹配结果中利用阅读理解技术截取答案。
这种二段式的阅读理解方案在当今数据文本爆炸式增长的时代,不用处理过长的文本,可以大幅减少计算成本,并且结果也比传统的长文本处理方式更加精确,目前已经被广泛地应用。
此外,问答匹配任务对于文本语义的挖掘需求非常精确,这样的语义理解能力在传统的自然语言处理任务例如文本蕴含、篇章关系分析等方向都有着重要意义。

综上所述,问答匹配是自然语言处理领域的一项核心的研究任务,提高问答匹配的性能能够从根本上提高智能问答系统,搜索引擎等应用的准确性和可靠性,并且问答匹配对于其他自然语言处理任务的研究和发展都有着重要意义。

\section{国内外研究现状}

问答匹配是文本匹配领域面向问答场景文本的分支,早期文本匹配方法主要为向量空间模型(Vector Space Model,简称 VSM)\cite{salton1975vector}和基于概率检索模型的BM25\cite{robertson1994some},由Salton等人和Stephen等人提出。
主要用于建模词汇层面的字面匹配问题,该类方法缺乏对文本语义的深度理解,无法有效利用文本中的上下文语义。
随着深度学习方法的兴起,基于神经网络构建的文本匹配模型能够建模句子整体语义的高维特征表示。
这类方法突破了传统匹配方法停留在字面匹配的局限性,将文本匹配方法引入了新的研究方向。
深度文本匹配模型按照模型处理结构和语义信息表征的不同可以分为两大类,包括基于表征学习的方式和基于交互的方式。
而数据增强(Data Augmentation)作为一种资源扩展和难样本增广的方法,广泛地用于提高深度文本匹配模型的语义表征能力和难样本解决能力。
此处从以下三个方面详细分析该任务的国内外研究现状:

\textbf{\songti (1)基于表征学习}

基于表征学习的方式集中于研究如何构建语义表示层,用于将文本表示为高维向量。
这类方法通常使用孪生网络结构(Siamese Network)构建神经网络,这是一种共享参数的双通道结构。
利用双通道分别对“句子对”的两个句子进行独立编码,将其语义信息转化为高维特征向量。
得到两个分别表征两个句子的语义信息后,继而使用相似度计算方法,例如余弦相似度、欧几里得距离等,计算文本间的语义关系。
He等人首先提出的基于深度网络的语义模型DSSM\cite{huang2013learning},最先利用孪生结构的全连接神经网络对文本提取语义特征。
并且在DSSM基础上将全连接神经网络改进卷积神经网络(Convolutional Neural Networks,简称CNN)作为特征提取器的CLSM\cite{shen2014latent},以及长短期记忆网络增强(Long Short-Term Memory,简称LSTM)的LSTM-DSSM\cite{palangi2014semantic}模型。
He等人提出了一种孪生CNN的表示学习结构\cite{he2015multi},他们的实验比较了余弦相似度、欧氏距离和元素级作差等方法在相似度关系计算上的效果。

基于表征学习的方式也有着难以判断语义焦点、无法利用关键上下文的语义信息等缺点。
为了能够有效地缓解上述的问题,文本中的关键上下文需要能够加强其表示权重。
例如,Santos等人提出的AP-BiLSTM\cite{santos2016attentive}是将注意力机制与LSTM相结合的表示学习模型,其验证了注意力机制聚焦语义的正面作用。
Wang等人\cite{wang2016sentence}将两个句子按照词级分解为相似和不同两个集合,并通过CNN提取相似特征和可区分特征。 
Lai等人则是关注到由于中文分词可能引起的歧义,提出了一种基于词格(word-lattice)的CNN模型(LCNs)\cite{lai2019lattice}。

随着预训练语言模型(例如BERT\cite{devlin2018bert})的成熟应用,其强大的语义编码能力为文本匹配任务带来了显著的提升。
前人应用预训练语言模型作为双通道的编码器,对句子对分别进行独立地编码。
例如,Lyu等人\cite{lyu2021let}同样关注到,尽管应用了预训练语言模型,但是中文匹配任务上出现的分词歧义问题仍然难以解决.
于是提出了外部知识网络HowNet\cite{dong2003hownet}和词格构建图结构,利用图神经网络进行编码。
Wang等人提出了DABERT\cite{wang-etal-2022-dabert}的双流注意力方法,利用亲和注意力和差异注意力分别构造了句子对的相似特征和差异特征进行融合,得到更加完善的句子表示。


\textbf{\songti (2)基于文本交互}
 
基于交互方式的研究认为,不管是“问题-答案”或是“问题-问题”,句子对之间都存在着某种联系,双通道的独立编码方式(基于特征学习的方式)没有很好地利用这一联系。
基于交互的方式认为建模语义编码的过程应当更早地进行两段文本的交互,以此挖掘文本之间的交互特征,这样的特征能够很好地找到文本对子之间的语言联系。
前人的实验证明,基于文本交互的模型方法能够更好地在文本之间双向地捕捉焦点信息,且在特征表示中凸显句子对中交互性较强的语义信息,这些语义信息往往起着关键性因素。
例如,Pang等人提出的MatchPyramid\cite{pang2016text}受到图像处理的启发,用文本对子之间的交互构建了相似度矩阵,并利用CNN网络作为特征提取器对相似度矩阵进行编码。
Gong等提出深层交互推理模型DIIN\cite{gong2017natural},其构造元素级高阶交互,并使用密集连接卷积网络提取特征。DIIN在深层交互的同时,一定程度上保留了原始特征信息。

同样,随着注意力机制的广泛应用,其作为一种理想的交互计算方法,也被广泛应用在交互式方法的研究中,并且有着良好的表现。
例如,Wang等人提出了“匹配-聚合”框架下的 BiMPM 模型\cite{wang2017bilateral},该模型进行双边多角度的匹配操作以获得交互信息。 
Chen等人提出的ESIM\cite{chen-etal-2017-enhanced}模型则在LSTM的基础之上进行交互式推理,其通过元素点积和减法来锐化交互信息。
受ResNet\cite{he2016deep}的启发,Kim等人提出了一种结合了循环神经网络RNN、残差链接以及注意力机制的方法DRCN\cite{kim2019semantic}。
上述的方法都在语义表征过程中,利用注意力机制在句子对之间进行语义交互,凸显了句间的特征联系,并将其更好地融入句子的语义表示。

预训练语言模型也因其优秀的特征提取能力和多层的注意力机制带来的优秀交互能力,在问答匹配的交互方法上取得显著成果。
预训练语言模型在文本匹配任务上的交互方式是,通过将句子对拼接,作为统一的输入进入模型,由注意力机制等方式进行信息交互。
例如,Zhang等人提出了一种基于 BERT 的关系学习网络 ($R^2$-Net)\cite{zhang2021making},其利用多粒度语言单元的交互关系,更深层地学习标签所带来的语义信息。
以及,Xu等人最新提出的ISG-BERT\cite{xu2022semantic}将两个句子的句法对齐结果和语义匹配信号集成到关联图中,以获得更细粒度的匹配过程。

\textbf{\songti (3)数据增强方法}

前人的研究表明,数据量小、样本难度低等问题容易引起模型的过拟合问题。
数据增强方法通过扩大训练数据、提高数据难度和质量等方式,达到优化模型性能的效果。
回译(Back Translation,简称BT)是一种最常见的数据增强方式,Yu等人\cite{yu2018qanet}利用英法回译,先将数据转换成法语,再回译成英语,利用翻译过程中的差异达到用不同表述复述原句的效果。
但是,回译策略受到专业名词、地方文化不同、语言之间的语用习惯差异大等问题的影响,许多内容难以准确的翻译,这也会导致回译数据产生语义偏移和阅读起来不自然等问题。
Wei等人提出了一种简单数据增强策略EDA\cite{wei2019eda}在原样本的文本上随机删除、插入、交换字符,以及进行同义词替换四种操作构造新的增强样本。
尽管EDA策略实现简单,效率极高,但是这样的随机操作容易产生有着严重语法错误、语义偏移、甚至完全不可读等情况。
将这样的样本加入训练数据,会导致模型训练过程有着极大的噪声,会导致其数据增强效果的下降。

为了缓解上述的问题,Jin等人提出的TEXTFOOLER\cite{jin2020bert}方法利用外部知识库WordNet提供同义词列表进行替换,再使用通用句子编码模型(Universal Sentence Encoder,简称USE)\cite{cer2018universal}监测替换后的句子和原句是否有较大的语义偏移,若有,则舍弃该样本。
同样基于WordNet的同义词替换策略,PWWS方法\cite{ren-etal-2019-generating}则提出了一种概率加权的词显着性方法来确定词替换顺序。
而Li等人提出的BERT-ATTACK\cite{li2020bert}方法,为了能够缓解同义词替换带来的不自然和语法错误的问题,利用语言模型的掩码预测任务(Masked Language Model,简称MLM)代替同义词替换,即将需要替换的词遮蔽,使用MLM任务生成该位置的候选词。

前人的研究,例如Morris等人\cite{morris-etal-2020-reevaluating}发现,基于同义词替换或是随机字符修改的方法容易引起语法错误,语义偏移,阅读不自然等问题。
前人也提出了其他方式的数据增强方法来改善这一问题。
例如,Gary等人\cite{garg2020tanda}提出的TANDA(Transfer and Adapt Pre-train Model,简称TANDA)利用了多段迁移学习的方式,在通用领域或是相似领域的数据上预先学习一部分知识,作为样本数据的扩充。
为了提高生成的效率和质量,前人开始尝试生成式的数据增强方法。
随着自回归(Auto-Regeressive)的生成式模型不断成熟,越来越多的工作开始以端到端(sequence to sequence,简称seq2seq)模型作为数据增强方法。
例如,Hou等人\cite{hou-etal-2018-sequence}提出了一种基于seq2seq模型的填槽式样本生成方法,而Kumar等人\cite{kumar-etal-2020-data}对自编码器的BERT以及自回归的GPT-2\cite{radford2019language}和BART\cite{lewis2019bart}等模型进行文本增强的对比。
基于seq2seq的生成式模型目前已经具备了生成符合人类语言习惯的优质文本的能力,在生成效果的多样性和流畅性上要优于同义词替换等方法,同时也具备良好的时间效率。


% 但是,生成式的方法受限于数据问题,并非一个普适的方法。该类方法需要至少满足两个条件中的一条,即数据本身具备可生成的条件或是调用外部同领域的其他可生成的数据集获取知识。
% 如何有效地利用生成式模型进行数据增强也成为了一个值得关注的研究方向。


\section{关键问题和研究难点}

由上述研究现状可知,基于深度学习技术的的问答匹配方法已经趋于成熟,取代了传统的匹配方案。
然而,由众多的研究工作表明,问答匹配技术仍然在各个方面有着不完善的缺点,在实际应用中仍面临着诸多挑战。
本节主要总结问答匹配任务的现有的关键问题及其原因,以及面向该问题的研究难点。

\subsection{关键问题}

% % 语言理解能力
% % 鲁棒能力(迷惑样本)
% 传统的基于字面匹配的匹配方案(如BM25)主要根据“文本对子”中的字面关键词进行匹配,无法关注文本的深层语义。
% 其方法通过对文本中关键词的词频和文档长度等因素进行加权计算,这导致了传统的匹配方法经常出现误差,无法准确识别文本之间的语义关系。
% 如,{\kai{“您今天晚上准备什么时候吃饭?”}}和{\kai“您今天晚上准备什么地方吃饭?”},这两句话具有字面有着高度的重复的关键词,传统的匹配方法往往难以找到关键点,而将其判断为同样语义的问句。

深度学习技术突破了传统匹配方法(如BM25等)只能通过字面关键词进行匹配的限制,能够将文本这样离散化的数据作为输入,通过神经网络编码得到语义信息的高维特征表示向量,再利用文本对子在向量空间的位置和距离等信息判断语义关系。
而近年的预训练语言模型,作为自然语言处理领域的一项突破,其出色的特征提取和上下文感知能力为问答匹配任务带来了显著的提升。

然而,受到微调的数据规模、难度、质量等方面的限制,仅仅依靠预训练语言模型对文本的语义理解程度远远不够,仍存在一些问题。
1)现有的答案选择方法,在非事实性问题的场景下,由于其答案通常呈现为较长的段落级回答,导致其难以准确提取语义核心,而性能表现较差。
于此同时,该类模型受到轻微的扰动(例如同义词替换),即会更改预测的答案,鲁棒性差。
2)在问句复述识别任务中,“同质异构”与“异构同质”类样本是模型最难以判断的两类困难样本。
这两类样本需要模型具有感知其精细语义的能力,当前的问句复述识别方法在这两类样本上的表现仍然欠佳。
3)现有的问句复述识别方法,忽视了问句意图的重要作用,导致其对于意图感知能力差。
而意图是否一致对于语义一致性的判断是十分重要的,如何增强模型对于意图的感知也是问句复述识别的一项关键问题。
因此,综上所述,强化模型对于关键语义的提取能力,以及提高模型对于问句意图的感知能力应该作为问答匹配任务的重点研究内容。


\subsection{研究难点}

基于现有问答匹配方法所存在的关键问题,我们将研究难点总结为以下三点:

\textbf{\songti (1)模型的语义提取能力弱,鲁棒性差}

预训练语言模型可以接收“文本对子”的输入,利用注意力机制,形成上下文相关的统一语义编码表示。
然而,在面向非事实性问题的答案选择任务中,候选答案呈现为较长的段落级回答,这使得全局注意力的噪声较大,无法准确提取段落的核心语义。
模型在训练过程中,面临着语义提取能力弱,并且容易受到噪声干扰等问题。
在不影响句子语义情况下的轻微扰动都常常引起模型预测结果的改变,例如,关键词的同义词替换,句子向量的细微扰动等。
这种不稳定性正是因为其关键语义的感知偏差导致的,模型在较长文本的场景下,无法抓住文本的核心语义信息,进而导致了抵抗不了轻微干扰的弱鲁棒性。
如何设计一种在非事实性答案选择任务场景下,设计一种增强模型关键语义提取能力和稳定性的训练方法,是一项重要的研究内容。


\textbf{\songti (2)“同质异构”和“异质同构”类样本难解}

在问句复述识别任务中,细微的语义差异会导致文本之间的语义关系发生很大的变化。
“同质异构”样本指两个文本具有相同的语义但是字面的表现形式差异较大,其实际关系应为“复述”,而模型由于其字面差异过大判断为“非复述”。
而“异构同质”样本指两个文本具有相近的字面表示形式,但其语义不同,其实际关系应为“非复述”,而模型由于其相似的字面关系判断为“复述”。
预训练语言模型由于其预训练任务与下游任务之间的差异,导致其对于精细的语义的感知并不敏感。
如例 1-1 所示的“异质同构”类样本,两个问句只有一字之差,编辑距离仅为1,但是却代表着不同的语义。
模型能够判断这类样本的前提是需要能够感知到这样关键的细微差别。
\begin{quotation}
    \noindent \textbf{\songti 例1-1}(“异质同构”问题对子样例)
    
    \noindent \textbf{问句1:} 淘宝邮政编码怎么\underline{填}?
    
    \noindent \textbf{问句2:} 淘宝邮政编码怎么\underline{改}?
    
    \noindent \textbf{编辑距离:} \underline{1} \qquad \textbf{关系:} \underline{非复述}
    
\end{quotation}

如例 1-2 所示的“同质异构”类样本,两个文本之间的表述差异较大(编辑距离为8),但是两者却有着相同的语义,这对于模型而言也是巨大的挑战。

\begin{quotation}
    \noindent \textbf{\songti 例1-2}(“同质异构”问题对子样例)
    
    \noindent \textbf{问句1:} 管理者怎样才能让员工服从?
    
    \noindent \textbf{问句2:} 怎样使员工服从领导?
    
    \noindent \textbf{编辑距离:}  \underline{8}  \qquad \textbf{关系:} \underline{复述}
    
\end{quotation}

然而,现有的问句识别数据集中,该类别的难样本在训练数据中占比极少,模型无法在训练过程中获得感知其精细语义的能力。
数据增强方法是缓解该问题的重要方法,但是现有的数据增强难以产出高质量的“同质异构”和“异质同构”样本,成为问句复述识别任务的研究难点之一。

\textbf{\songti (3)问句意图感知能力弱}

问句意图作为识别问句语义一致性的重要信息,但是现有的问句复述识别方法忽略了这一问题。
如例 1-3 所示,问句1的问句意图是询问人,即为“Who”类型的问题,而问句2的问句意图是“What”类型,即在询问定义。
\begin{quotation}
    \noindent \textbf{\songti 例1-3}(不同问句意图的非复述问题对子样例)
    
    \noindent \textbf{问句1:} \underline{Who} exactly is successful?

    \noindent $<$\textit{译文:到底谁可以称为一个成功人士?}$>$
    
    \noindent \textbf{问句2:}  \underline{What} exactly is success?

    \noindent $<$\textit{译文:到底什么可以称为成功?}$>$
    
    \noindent \textbf{关系:} \underline{非复述}
    
\end{quotation}

两者的虽然字面相近,但是问句意图却是完全不同的,这一点从翻译成英文后的疑问词可以明显的发现。
提高模型对于问句意图的感知能力,能够使其对于问句对子之间的语义关系判断更加精确。
现有的模型方法很少从问句意图的角度出发,因此,如何提高模型对于问句意图的感知能力是当前研究的一个难点。


\section{研究内容与组织结构}

针对上述问答匹配任务的关键问题及研究难点,本文从提高模型稳定性,对于文本的关键精细语义提取能力,以及增强模型问句意图感知能力等多个角度出发,提出了三个研究工作。
本节对其研究内容,以及论文的整体组织结构安排进行概述。

\subsection{研究内容}

本文研究内容具体可分为以下三个模块:

% \textbf{\songti (1)基于多粒度交互推理的答案选择方法研究}
\textbf{\songti (1)面向非事实性答案选择任务的双粒度对抗训练方法研究}

本文前期研究显示现有的答案选择方法在非事实性场景下,受到轻微噪声影响即会改变预测结果。
其根本原因是关键语义提取能力弱,受到较长文本的噪声干扰下,模型无法聚焦关键语义,因此稳定性差。
对抗训练方法作为数据增强方法的一种,具有生成困难样本进行数据增强,提高模型性能和稳定性的功能。
但是前沿的对抗训练方法往往集中于对于样本进行单一粒度的扰动(字符级,词级,句子级)。
这种方式生成的对抗样本质量低,扰动模式单一,容易被模型捕获。
为此,本文提出了一种基于词级和句子级的双粒度的扰动方法探测模型弱点(字符级扰动会导致不可读的问题),并对其弱点生成对抗样本,加入训练数据中供模型学习。
此外,我们还设计了一种样本过滤方法,筛选出具有挑战性的高质量对抗样本供模型训练,以这种方式提高模型的稳定性和关键语义提取能力。
% 其在BERT编码信息之上再次执行多粒度卷积编码,并且将多粒度交互信息与原始分类特征融合,形成蕴含关键线索的精细语义表示。
% 其二,候选答案中不同句子与问题的关联性具有显著差异,但现有模型的训练过程,并未考虑子句的权重差异。
% 为此,本文提出了一种句子级的损失优化策略,侧重提升关键语句在答案选择过程中的作用。

\textbf{\songti (2)面向问句复述识别的生成式增强学习方法研究}

前沿的问句复述识别模型仍然难以应对“同质异构”和“异质同构”类的困难样本。
这两类样本需要模型具有精细的语义感知能力,但是训练数据中该类数据的比例难以在预训练语言模型的微调阶段给予其足够的支撑。
而数据增强技术是缓解这一问题的有效途径,但是如何高效地生成高质量的这类困难样本,成为了该任务上数据增强技术的难题。
传统的数据增强技术,例如回译,对抗样本等方法,存在着生成效率低的问题,并且其生成的样本也有着缺乏多样性,甚至存在着不可读,不自然,以及与原样本语义偏移较大等严重的质量问题。
为此,本文提出了一种基于生成式的数据增强方法,其利用问句复述的语料构造出“同质异构”和“异质同构”的两个子数据集,并用于训练两个独立的端到端的生成模型用于数据增强,生成效率和质量都优于现有的主流方法。
此外,本文还设计了一个错题本式的增强训练方法,能够高效地增强模型在困难样本上的表现,增强模型对于其细微语义需求的感知能力。
% 数据增强策略能够有效扩充蕴含精细语义关系的数据样本,在一定程度上能够提高模型的任务适应性。
% 传统的简单数据增强策略以及回译策略能够高效地扩充数据数量,但是其仅能构造与原样本语义相同的数据,并且无法保证构造样本的语法正确性和标签一致性(即样本不合格),这些包含错误知识的数据对模型产生负面影响。
% 本文提出一种基于生成模型的定向数据增强策略(Directional Data Augmentation,简称DDA),在生成模型的输入中添加定向标签,引导其生成期望的复述句或非复述句。
% 此外,本文设计了一种多模型集成的标签投票机制,并用其修正增强样本的潜在标签错误,以此提高扩展数据的可靠性。

\textbf{\songti (3)基于问句意图感知强化的问句复述识别方法研究}

问句意图的一致性是判断问句语义一致性的重要条件。
相同语义的问句对子,其意图也应该是一致的。
而意图不同的问句,其语义应为不相同。
而当前的问句复述匹配方法缺乏对于问句意图的感知研究,导致模型在字面语义相近的情况下忽略了意图的一致性而误判。
尤其在中文场景下,由于中文复杂的语法和语用习惯,对于现有的深度模型来说,象征着问句意图的关键词往往难以抽取,可能存在于句子的各个位置。
本文提出了一种基于条件变分自编码器的问句意图强化方法,其意图提取模块能有效地提取问句意图,作为额外输入信息强化其语义表示,使得模型能够在问句意图信息的基础之上判断语义一致性。
此外,本文在上述模型的基础上提出了一种基于多任务方式的效率优化方法,能够在牺牲微小的性能的情况下大大提高其实际应用的潜力。
% 鲁棒性是反应模型能力的一项重要评价指标,缺乏鲁棒性的模型在现实应用中的性能往往并不理想。
% 目前一些研究已经开始关注模型鲁棒性的研究,但大多只针对某种特定场景下的数据,或是只使用了少量的数据变化方法,缺乏综合性的评估方案。
% 为了系统地评估问题复述识别模型的综合能力,本文构造了CQM$_{robust}$中文评估数据集,其包含3大类语言特征、13个测试子类,共计32种蕴含不同语言学特征的数据测试项。
% 同时,CQM$_{robust}$数据集中的所有问题均源于百度搜索日志中的真实问题,与实际应用场景匹配。
% 实验结果证明,CQM$_{robust}$比传统数据集难度更大,更具有挑战性,且能够对按照语言学现象对模型的能力进行详细评估,这有助于诊断不同模型的优点和缺点,为模型的优化方案提供参考。


\subsection{论文组织结构}

本文共分为六个章节,论文的组织结构和各个章节的主要内容如下:

第一章 \quad 绪论。本章从研究背景和研究意义对问答匹配任务进行了详细介绍,并且分析了国内外的研究现状,总结了目前研究的关键问题和研究难点。最后,阐述了本文的研究内容和论文组织结构。

第二章 \quad 任务定义及评价方法。本章介绍了问答匹配的任务定义、实验数据语料以及主流的评价指标。

% 第三章 \quad 基于多粒度交互推理的答案选择方法研究。本章首先介绍了该方法的动机,然后重点阐述了该方法的模型架构和具体实现细节。最后,本章通过实验验证所提方法的有效性。

% 第四章 \quad 面向问题复述识别的定向数据增强方法。本章首先介绍该策略的动机,然后介绍了定向数据增强策略据的实现细节,最后给出实验结果及分析。

% 第五章 \quad 面向问题复述识别的模型综合能力鲁棒性研究。本章首先分析了鲁棒性对于问题复述识别模型的必要性,其次介绍了CQM$_{robust}$数据集的结构及其构造细节。
% 最后,使用多个基线模型进行实验及分析。

第六章 \quad 总结与展望。本章对本文的工作进行总结,并展望后续工作。



