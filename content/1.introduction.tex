\chapter{绪论}

本章将对问答匹配任务展开详细介绍。
首先,本章阐述了问答匹配任务的研究背景及意义。
% 其次,本章介绍了该任务的国内外研究现状。
接着,本章总结了前人的科研工作,分别从基于表示、交互和数据增强三个方面,详细介绍了国内外研究现状。
然后,本章分析并总结了问答匹配领域现有的科学问题及研究难点,并针对以上问题和难点,阐述了本文的主要研究内容。
最后,本章简述了全文的组织架构。

\section{研究背景}

%% 问答系统的重要性
% 现有互联网信息获取的主要渠道为搜索引擎,其根据用户的提问返回一系列相关的网页链接。
% 人们通常在互联网上获取信息的主要渠道为搜索引擎,其根据用户的提问返回一系列相关的网页链接。
% 近年来,互联网技术高速发展,网络数据也呈爆炸式增长,
% 这导致用户需要花费大量的时间与精力从繁杂冗余的搜索结果中获取有用信息。
% 同时,网络上数据的质量参差不齐,传统搜索引擎无法有效甄别数据的合理性。
% 在此背景下,智能问答系统应运而生。

%% 问答系统的细分及匹配技术的重要性
问答匹配(Question Answer Matching)是自然语言处理(Natural Language Processing,简称NLP)领域重要的研究方向。
目前已在多种场景,如问答系统、客服机器人、语音助手等,得到广泛应用,当前主流的搜索引擎(百度、谷歌等)也已引入基于深度学习的问答匹配技术,以提供更加精准可靠的搜索结果。
问答匹配任务可以通俗地理解为面向问答场景文本的语义匹配任务。
作为问答匹配的子任务,答案选择(Answer Selection)以及问题复述识别(Question Paraphrase Identification)从语义匹配的不同角度为实际应用提供服务。
答案选择任务需要模型根据给定问题与候选答案的相似性,对候选答案进行排序,选择出相关性较高的答案进行返回。
问题复述识别任务则需要模型识别问句对子之间的语义一致性,即两个问句的意思是否一致,从而从已知答案的庞大问答对子中找出与用户提问最相关的问题,并返回其答案。
从答案选择以及问题复述识别任务的定义可知,两者分别从“问题与答案”的语义相关性以及“问题与问题”的语义一致性的两个角度,为智能问答系统,搜索引擎等实际应用提供支持。
其对知识获取、信息检索等领域也具有重要影响力。因此,问答匹配技术的研究具有十分重要的意义。

% 问答匹配的发展(预训练) 及缺点
随着预训练语言模型的发展,其优秀的特征提取和语义编码能力在NLP领域带来了显著的提升,其中也包括问答匹配任务。
问答匹配技术的研究一直备受关注,基于预训练语言模型的深度学习方法也成为了问答匹配任务的主流方法。
但是尽管研究的方法不断改进,却始终存在着两个挑战性的问题:其语义表征能力难解困难样例和问句意图感知能力弱。
首先,困难样本难解的根本原因是模型的语义表征不准确,没有精细地感知到语句中细微且关键的要素。
利用困难样本生成技术,例如对抗攻击方法,构造具有迷惑性的样本,加入模型的训练,是提高语义感知和表征能力的有效方法。
研究如何高效且高质量地生成困难样本对于提高问答匹配模型性能具有着重要的意义。
其次,问句意图感知能力弱也是当前问答匹配技术需要解决的另一个关键问题。
对于问句来说,意图信息是十分关键的。
例如:“您今晚什么时候吃饭?”和“您今晚什么地方吃饭?”,尽管字面十分相似,但是前者的意图是对时间提问,而后者是对地点提问。
仅仅凭借预训练语言模型的方法容易被相近的字面所迷惑,而忽视了更加关键的意图信息。


% 但是,这并不意味着模型在实际应用中仍能保持其高性能。
% 在真实的语言场景下,一些简单且易于回答的问题,最前沿的模型却会给出错误回答。
% 在实际应用中,问答匹配技术仍处于起步阶段,需要不断革新,使问答系统更加高效、准确。

\section{研究意义}
问答匹配包括答案选择以及问句复述识别两个子任务。问答匹配无论是对实际业务应用或是自然语言处理的科学研究都有着重要的研究意义。
此处从以下三个方面阐述其研究意义:

\textbf{\songti(1)有广泛的落地应用场景}
问答匹配任务有着广泛的实际业务需求应用,例如搜索引擎、自动问答、智能客服等领域。
其需要系统能够充分理解用户所输入问句的语义,并从已有的数据源中匹配出相关性最佳的结果。
现有的主流搜索引擎中已经引入基于深度学习的问答匹配技术。
用户输入问题后,搜索引擎内部利用问答匹配和机器阅读理解等技术,解析出精确的答案作为第一条结果返回,极大提高用户搜索体验。
此外,例如信息检索技术可以归结为是用户输入的查询与文档资源的语义匹配,自动问答系统则是通过用户查询与已有候选答案或者是已有“问答对子”的问句进行匹配。
包括众多商家如淘宝、京东等都接入了智能客服系统,用户解决海量的客户问题及反馈。
利用这种方式可以高效的解决用户的常见问题,降低运营成本,提升用户体验,这也是问答匹配技术的一项成熟的应用。
问答匹配是自然语言理解中的核心问题,其泛用性成为许多前沿智能系统的基础技术。


\textbf{\songti(2)对问答系统的重要意义}
传统的搜索引擎根据用户查询返回相关的网页,这样的方式导致用户无法快速便捷地获取关键信息。
而由深度学习技术兴起的自动问答系统以“一问一答”的智能形式,给予用户良好的交互体验。
通过这种方式可以精确定位用户意图,从而快速高质量地满足用户的信息需求,达到降低成本,提高用户体验的效果。
答案选择技术提供了“问题-答案”相关性语义的支持,根据用户所输入的问句的语义信息,从已有的候选答案数据中匹配出最佳结果。
例如,该技术应用于基于知识库的问答系统(Knowledge Base Question Answering, 简称KBQA),为用户输入的问题召回相关性最高的知识,以用于模型结合问句以及知识之后产出答案。
问题复述识别则从“问题-问题”之间的语义一致性出发。根据用户输入的目标问句,从已知答案的候选“问句-答案”对子中,匹配出语义相同或高度相似的复述问句,将其答案返回给用户。
例如对于常用的检索式问答系统,以及主流的搜索引擎(如百度,谷歌等)而言,问句复述识别是实现该类系统的关键。
因此,提高问答匹配技术对于自动问答,信息检索等智能系统的准确性和可靠性的具有非常重要的意义。


\textbf{\songti(3)辅助其他自然语言处理任务}

问答匹配任务可以通俗地理解为面向问答场景文本的语义匹配任务,其任务的核心在于理解句子对的语义,并且根据任务场景作出相应的判断。
其语义理解和匹配的能力为自然语言处理领域的其他任务提供了强大的支持。
例如,在常识问答(Commonsense QA)任务中,需要模型在获取到问题之后,检索与问题相关的常识知识。借由这些知识的语义交互,模型才能产出该问题的答案。
问答匹配技术能够根据问题,有效地检索出知识库中与之相关的常识信息,返回给模型。问答匹配技术的召回的知识是否准确和可靠,也成为了常识问答模型是否能够准确回答问题的关键。
此外,在开放域阅读理解(Open-domain Machine Reading Comprehension)任务中,首先需要借助问答匹配技术在庞大的文档资源中,匹配出较为相关的段落,继而在匹配结果中利用阅读理解技术截取答案。
这种二段式的阅读理解方案在当今数据文本爆炸式增长的时代,不用处理过长的文本,并且大幅减少计算成本,结果也比传统的长文本处理方式更加精确,目前已经广泛地应用。

综上所述,问答匹配是自然语言处理领域的一项核心的研究任务,提高问答匹配的性能能够从根本上提高智能问答系统,搜索引擎等应用的准确性和可靠性,并且问答匹配对于其他自然语言处理任务的研究和发展都有着重要意义。

\section{国内外研究现状}

问答匹配是文本匹配领域的一个分支,包括答案选择和问题复述识别两个主要的子任务。
传统的文本匹配方法,如Salton等提出的向量空间模型(Vector Space Model,简称 VSM)\cite{salton1975vector}和Stephen等提出的基于概率检索模型的BM25\cite{robertson1994some},主要解决词汇层面的匹配问题,其无法有效利用文本中的语义现象。
基于深度学习方法构建的文本匹配模型能够突破传统匹配方法存在的语义局限性,在不同匹配任务上均有着良好表现。
深度文本匹配模型按照语义信息处理方式的不同通常可以分为两大类,包括基于表示学习的方式和基于交互的方式。
此外,一些研究开始尝试采用数据增强(Data Augmentation)的手段优化深度文本匹配模型。
此处从以下三个方面详细分析该任务的国内外研究现状:

\textbf{\songti (1)基于表示的方式}

基于表示的方式侧重对文本语义表示层的构建,表示层将文本中的语义信息转化为分布式向量,继而使用余弦相似度等算法计算文本间的匹配程度。
比如,He等提出的深度结构语义模型DSSM\cite{huang2013learning},首次使用全连接神经网络进行语义特征提取,并在DSSM基础上提出CLSM\cite{shen2014latent}和LSTM-DSSM\cite{palangi2014semantic}模型,通过卷积神经网络(Convolutional Neural Networks,简称CNN)\cite{krizhevsky2012imagenet}以及长短期记忆网络强化了文本语义表示。
基于表示方式的核心是学习文本的语义特征,但是容易失去语义焦点、发生语义偏移,从而难以识别和利用支配性更强的上下文语义信息。
从而,在学习语义表示的过程中融入注意力机制,加强关键上下文的表示权重,成为近期解决上述问题的重要手段之一。比如,Santos等提出的AP-BiLSTM\cite{santos2016attentive}是一种在长短记忆网络LSTM之上结合注意力计算的表示表示学习模型,其实验验证了注意力机制对整体性能的正面作用。

\textbf{\songti (2)基于交互的方式}

基于交互方式的模型设计初衷源自于问题和答案固有的交互性。交互性也被看做是语义的近似性和关联性。从而,相关工作的研究认为:表示学习过程应当更早地模拟两段文本的交互关系,进而挖掘文本交互后的模式特征,借以综合得到文本之间的匹配度。前人工作表明,基于交互的神经网络能够在文本之间双向地捕捉焦点信息,且在表示模型中凸显交互性较强的语义信息。
比如,Pang等提出的匹配金字塔模型MatchPyramid\cite{pang2016text}即是代表性工作之一,其借鉴图像匹配的交互表示思想,构建了文本之间相似度矩阵,并在此基础上利用CNN对相似度矩阵进行特征提取。Gong等提出深层交互推理模型DIIN\cite{gong2017natural},其构造元素级高阶交互,并使用密集连接卷积网络提取特征。DIIN在深层交互的同时,一定程度上保留了原始特征信息。
近期,基于注意力的交互计算方法也被相继提出,并在文本匹配任务中显示了可靠的性能。
比如,Wang等提出的双边多角度句子匹配模型BiMPM\cite{wang2017bilateral}和Kim等提出的密集递归交互模型DRCN\cite{kim2019semantic},都在表示阶段通过注意力机制进行语义交互,丰富了句内与句间的语义特征。
此外,Ru ̈ckle ́提出的COALA\cite{ruckle2019coala}模型通过聚合运算,泛化了问题与候选答案之间的交互式表示,从而在一定程度上缓解了高复杂度交互模型对大规模训练数据的过度依赖。

目前,预训练语言模型已经广泛应用到文本匹配任务中,并取得显著成果。
这类模型在全新的任务场景中,能够复用且微调自注意力和联合编码机制,形成适应性更强的注意力和交互计算。
预训练语言模型在文本匹配任务上可以看作是基于表示和基于交互两种方式的融合,其在语义表示的基础上进行了交互加强。
Nogueira\cite{nogueira2019passage}等将预训练语言模型BERT应用于段落级匹配任务中,在此基础上Mass等提出的BERTlets\cite{mass2019study}方案,通过合页损失及段落分割策略,优化了长文本的答案选择性能。

\textbf{\songti (3)基于数据增强的方式}

近期,相关研究尝试通过扩大训练数据规模、提高数据质量等手段优化深度模型。
Yu等将回译策略(Back Translation,简称BT)\cite{yu2018qanet}用于特定自然语言处理任务中,将英文问答数据翻译为法文后,再回译成英文数据,以此扩充问答数据量,提高机器阅读理解模型的表现。
回译策略构造的增强样本中包含期望扰动(不改变句子原意),有利于提高目标模型的泛化能力。但是,受限于现有机器翻译模型能力的制约,翻译过程中加入期望扰动的同时,也会导致回译数据产生语义偏移等问题。
Wei等提出的简单数据增强策略(Easy Data Augmentation,简称EDA)\cite{wei2019eda}在原样本的基础上采用随机删除、插入、交换、同义词替换四种操作构造新的增强样本。EDA策略易于实现,对小规模数据集效果尤为显著。但是随机的操作对语法、句法造成诸多的不确定性,导致增强样本极易出现不可读以及语义偏移等情况。
Jin等提出的TEXTFOOLER\cite{jin2020bert}方法,在同义词替换后,借助谷歌开源的通用句子编码模型(Universal Sentence Encoder,简称USE)\cite{cer2018universal}剔除语义偏移明显的增强样本,一定程度上能够提高增强样本质量。
Li等提出的BERT-ATTACK\cite{li2020bert}方法,将原样本中的部分词遮盖掉,借助语言模型掩码预测任务得到遮盖词的候选替换词。这种结合上下文预测候选词的方案能够有效缓解增强样本中的语法错误问题。
此外,Gary等\cite{garg2020tanda}构造出用于文本匹配任务的大规模高质量迁移学习通用数据集ASNQ,并提出通过迁移学习调节预训练语言模型的方法(Transfer and Adapt Pre-train Model,简称TANDA),使得预训练语言模型在多个文本匹配任务中的表现得到大幅提升。


\section{关键问题和研究难点}

由上述研究现状可知,基于深度神经网络的问答匹配研究已经取得了一定进展。
然而,大量的科研成果尚处于学术研究阶段,问答匹配技术在实际应用中仍面临着诸多挑战。
本节主要介绍问答匹配的关键问题以及研究难点。

\subsection{关键问题}

% 语言理解能力
% 鲁棒能力(迷惑样本)
传统的基于词共现的问答匹配方案(如:VSM、BM25)主要根据“文本对子”中关键词的共现程度判定两者的语义关系。
这种方式忽略了文本的语义信息,模型无法理解词意、句意以及句法构成。
例如,{\kai{“什么水果脂肪含量高”}}和{\kai“什么水果脂肪含量低”}两个字面高度重复但却明显具有不同语义的问句,基于词共现的问题复述识别模型往往将两者判定为复述关系。

基于深度神经网络的问答匹配方法能够突破传统方法的语义局限性,其使用词的向量化表示代替词本身作为信息交互的基本单位,利用“文本对子”中词与词在高维空间中的位置关系感知两者语义异同。
当前的预训练语言模型在训练过程中采用语言掩码策略,使词向量在不同语句中能够动态表达语义信息,有效解决一词多义等问题,进一步提高了模型在问答匹配任务上的性能。
然而,真实场景下的问答往往包含大量的噪声干扰,现有的模型对文本的语义理解程度远远不够,仍存在一些问题。
例如,答案选择模型在处理非事实性问题(答案较长,一般由多个句子或段落构成)时,容易被答案中的冗余信息干扰,错失关键线索。
同时,日常生活中常见的错别字、特定表达等也都会对模型的判定结果造成极大干扰。
综上所述,优化模型的语义处理能力和提高模型的鲁棒性应该作为问答匹配任务的重点研究内容。


\subsection{研究难点}

通过分析问答匹配任务的研究现状及关键问题,将研究难点总结为以下三点:

\textbf{\songti (1)注意力交互容易忽略精细语义线索}

预训练语言模型蕴含的自注意力机制能够在“文本对子”之上,形成统一的语义编码表示。
然而,现有的注意力机制多为全局注意力机制,会对上下文中所有词进行注意力交互,从而导致交互信息噪声过大。
尤其在文本较长的非事实性答案选择任务中,仅依赖全局注意力机制进行交互容易忽略关键词块、短语和子句的独立语义信息表示,使得文本在匹配过程中容易错失精细粒度语义相关性的感知。
因此,如何获取不同粒度的语义交互信息,捕获关键的精细语义线索,是当前的研究难点之一。

\textbf{\songti (2)缺少蕴含精细语义表达的样本}

预训练语言模型善于生成通用的语义表示,对特定任务中精细的语义表示需求并不敏感。
在问题复述识别任务中,样本的复述关系往往取决于极为微妙的语义差异。
如例 1-1 所示,模型需理解{\kai“iphone6”}与{\kai“iphone6x”}之间的细小差异,才能准确判断两者为非复述关系。
\begin{quotation}
    \noindent \textbf{\songti 例1-1}(非复述关系问题样例)
    
    \noindent \textbf{问句1:} What is the price of \underline{ipone6}?
    
    \noindent $<$\textit{译文:\underline{ipone6}手机的价格是多少?}$>$
    
    \noindent \textbf{问句2:} What is the price of \underline{ipone6x}?
    
    \noindent $<$\textit{译文:\underline{ipone6x}手机的价格是多少?}$>$
    
\end{quotation}

然而,现有的问题复述识别数据集缺少上述蕴含细微语义差异的样本,
导致预训练语言模型在进行微调后,仍难以作出准确判断。
所以,如何扩充高质量样本以提高模型对于精细语义的感知能力,成为当前问题匹配任务的研究难点之一。

\textbf{\songti (3)模型的鲁棒能力不明确}

随着自然语言处理技术的不断发展,深度学习模型在问答匹配领域的表现正在稳步攀升。
事实上,在真实场景中模型的语言理解水平并未达到理想水平,一个简单且细微的改动就能使模型失效,诸如此类的例子屡见不鲜。
如例 1-2 所示,将问题{\kai“北京到上海的距离有多远”}中的{\kai“北京”}与{\kai“上海”}位置调换,一些模型就会将两者判定为非复述关系。
\begin{quotation}
    \noindent \textbf{\songti 例1-2}(复述关系问题样例)
    
    \noindent \textbf{问句1:} How far is the distance from \underline{Beijing} to \underline{Shanghai}?
    
    \noindent $<$\textit{译文:\underline{北京}到\underline{上海}的距离有多远?}$>$
    
    \noindent \textbf{问句2:} How far is the distance from \underline{Shanghai} to \underline{Beijing}?
    
    \noindent $<$\textit{译文:\underline{上海}到\underline{北京}的距离有多远?}$>$
    
\end{quotation}

现有的研究缺乏模型鲁棒性的重视和深入研究,仅关注模型在特定测试语料上的评测结果,忽略了其在实际场景中的应用能力。
鲁棒性是评价模型能力的一个重要指标,可用于评估模型在面对细微变动时,能否保持判断的准确性,也是模型泛化能力的体现。
因此,如何合理评估问题匹配模型的鲁棒性是当前研究的一个难点。


\section{研究内容与组织结构}

针对上述问答匹配任务的关键问题及研究难点,本文从提高模型对精细语义的感知能力以及综合评估模型的鲁棒性多个角度出发,针对性的提出了解决方案。
下面主要介绍本文的研究内容和论文的组织结构。

\subsection{研究内容}

本文主要针对问答匹配模型的精细语义感知能力及鲁棒性进行研究,对现有的预训练语言进行优化,
提升其综合能力。本文研究内容具体可分为以下三个模块:

\textbf{\songti (1)基于多粒度交互推理的答案选择方法研究}

本文前期研究显示现有的答案选择方法在如下两方面尚存在提升的空间:
其一,不同粒度的句子成分的语义表示,皆有助于预测问题与答案的局部语义关联性,然而注意力机制容易忽略精细粒度的语义线索。
因此,本文提出一种多粒度交互式推理网络(Multi-granularity Interactive Inference Net-work,简称MIIN),
其在BERT编码信息之上再次执行多粒度卷积编码,并且将多粒度交互信息与原始分类特征融合,形成蕴含关键线索的精细语义表示。
其二,候选答案中不同句子与问题的关联性具有显著差异,但现有模型的训练过程,并未考虑子句的权重差异。
为此,本文提出了一种句子级的损失优化策略,侧重提升关键语句在答案选择过程中的作用。

\textbf{\songti (2)面向问题复述识别的定向数据增强方法}

现有的问题复述识别模型对特定任务中精细的语义表示需求并不敏感。
预训练语言模型的微调阶段能有效提高模型对任务相关的细微语义感知力,但其极大依赖于训练数据的多样性与可靠性。
数据增强策略能够有效扩充蕴含精细语义关系的数据样本,在一定程度上能够提高模型的任务适应性。
传统的简单数据增强策略以及回译策略能够高效地扩充数据数量,但是其仅能构造与原样本语义相同的数据,并且无法保证构造样本的语法正确性和标签一致性(即样本不合格),这些包含错误知识的数据对模型产生负面影响。
本文提出一种基于生成模型的定向数据增强策略(Directional Data Augmentation,简称DDA),在生成模型的输入中添加定向标签,引导其生成期望的复述句或非复述句。
此外,本文设计了一种多模型集成的标签投票机制,并用其修正增强样本的潜在标签错误,以此提高扩展数据的可靠性。

\textbf{\songti (3)面向问题复述识别的模型综合能力鲁棒性研究}

鲁棒性是反应模型能力的一项重要评价指标,缺乏鲁棒性的模型在现实应用中的性能往往并不理想。
目前一些研究已经开始关注模型鲁棒性的研究,但大多只针对某种特定场景下的数据,或是只使用了少量的数据变化方法,缺乏综合性的评估方案。
为了系统地评估问题复述识别模型的综合能力,本文构造了CQM$_{robust}$中文评估数据集,其包含3大类语言特征、13个测试子类,共计32种蕴含不同语言学特征的数据测试项。
同时,CQM$_{robust}$数据集中的所有问题均源于百度搜索日志中的真实问题,与实际应用场景匹配。
实验结果证明,CQM$_{robust}$比传统数据集难度更大,更具有挑战性,且能够对按照语言学现象对模型的能力进行详细评估,这有助于诊断不同模型的优点和缺点,为模型的优化方案提供参考。


\subsection{论文组织结构}

本文共分为六个章节,论文的组织结构以及各个章节的主要内容如下:

第一章 \quad 绪论。本章首先介绍问答匹配任务的研究背景和意义,然后分析了该任务的国内外的研究现状,总结了当前的关键问题和研究难点。最后,介绍了本文的研究内容和论文组织结构。

第二章 \quad 任务定义及评价方法。本章介绍了问答匹配的任务定义、实验数据语料以及主流的评价指标。

第三章 \quad 基于多粒度交互推理的答案选择方法研究。本章首先介绍了该方法的动机,然后重点阐述了该方法的模型架构和具体实现细节。最后,本章通过实验验证所提方法的有效性。

第四章 \quad 面向问题复述识别的定向数据增强方法。本章首先介绍该策略的动机,然后介绍了定向数据增强策略据的实现细节,最后给出实验结果及分析。

第五章 \quad 面向问题复述识别的模型综合能力鲁棒性研究。本章首先分析了鲁棒性对于问题复述识别模型的必要性,其次介绍了CQM$_{robust}$数据集的结构及其构造细节。
最后,使用多个基线模型进行实验及分析。

第六章 \quad 总结与展望。本章对本文的工作进行总结,并展望后续工作。



