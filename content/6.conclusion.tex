\chapter{总结与展望}

\section{工作总结}

问答匹配是文本匹配领域的主要研究方向之一,
根据数据源的不同通常将其划分为答案选择和问题复述识别两个子任务。
答案选择任务关注“问题与答案”间的语义关系,根据目标问题从大规模候选答案中挖掘相关的正解,用以优化智能问答场景下召回相关答案的质量。
问题复述识别关注“问题与问题”间的语义关系,根据目标问题从候选问题集合中寻找同义问题(已知答案),并将其答案作为目标问题的正解反馈给用户。
目前预训练语言模型等深度模型在问答匹配领域已经取得一定进展,
然而在该任务中仍存在一些有待解决的问题,包括模型对精细语义不够敏感以及缺乏鲁棒性。
本文针对上述问题,提出一种基于精细语义感知和鲁棒性诊断的问答匹配方法,具体地工作总结如下:

\textbf{\songti (1)基于多粒度交互推理的答案选择方法研究}

本文发现不同粒度的语义成分,皆有助于预测问题与答案的局部语义关联性,形成多粒度的关联线索。
所以,本文提出一种多粒度交互式推理网络MIIN,其基于BERT编码特征二次进行多粒度卷积编码,
并且将多粒度交互信息与原始分类特征融合,保留问题与答案交互过程中的精细语义线索。
此外,长文本候选答案中不同子句与问题的关联性具有明显差异,
本文设计一种子句损失优化策略,侧重提升关键语句在答案选择过程中的作用。

\textbf{\songti (2)面向问题复述识别的定向数据增强方法}

本文提出一种定向数据增强策略DDA,用于提高问题复述识别模型对问题精细语义的感知能力。
其通过在生成模型UNILM的输入中添加定向诱导标签,生成期望的复述句或非复述句。
相较于传统的数据增强方案,DDA能有效降低文本的语法错误率,并且产出的样本质量更高、难度更大。
DDA为预训练语言模型的微调扩充了大量蕴含精细语义关系的问题对子样本,有效提高了模型对细微语义感知。

\textbf{\songti (3)面向问题复述识别的模型综合能力鲁棒性研究}

鲁棒性是反应模型能力的一项重要评价指标,能够检验模型在面对微小变化时的稳定性,也是模型泛化能力的体现。
本文构造了一个面向中文问题复述识别模型的评估基准CQM$_{robust}$,其包含3大类语言特征、13个测试子类,共计32个测试项。
CQM$_{robust}$中的所有问句均来自于百度搜索日志中的真实问题,能够更加客观反映模型的实际应用能力。
实验结果证明,CQM$_{robust}$更具有挑战性,且能够对按照语言学现象对模型的鲁棒能力进行详细评估。

\section{工作展望}

本文对问答匹配技术展开研究,提出一种基于精细语义感知和鲁棒性诊断的问答匹配方法,
在一定程度上解决了当前问答匹配领域存在的难点问题。
然而,其中仍存在可以改进的地方:

\textbf{\songti (1)优化特征融合的方式}

本文所提出的多粒度交互式推理方法中,直接采用拼接的方式将多种粒度的交互特征以及原始的分类特征进行融合。
然而,不同粒度特征信息的权重不同。
因此,在未来的工作中,可以考虑使用可训练的参数将不同粒度的信息进行加权融合,让模型在训练过程中对参数进行自我调节。

\textbf{\songti (2)生成更加多样化的问句}

本文所提出的定向数据增强策略,通过给定诱导标签能够控制生成模型问句的类型(复述或非复述)。
然而,诱导标签中包含的信息量过于单一,生成的问句中仍保留有绝大部分的原始问句信息,从而导致生成问句的语义大多都倾向于与原始问句语义近似。
因此,在后续工作中,应当考虑在生成过程中引入其他附加信息或计算操作,用以扩大生成问句与原始问句的差异,使生成的问句更加多样化。

\textbf{\songti (3)扩增语料的语种类型}

本文所构造的CQM$_{robust}$能够系统化的评估问题复述识别模型的鲁棒能力,并且其所有样本均为自然问句。
然而,CQM$_{robust}$中仅包含符合中文语言特征的样本,只能对中文问题复述识别模型作出详细诊断。
因此,在后续工作中,可以尝试对英文等语言现象进行分析,在数据集中加入其他语种的样本,
使得CQM$_{robust}$不仅能评估多种语言的问题复述识别模型,甚至能评估模型的跨语言理解能力。

